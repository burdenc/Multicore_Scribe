%
% This is the LaTeX template file for lecture notes for EE 382C/EE 361C.
%
% To familiarize yourself with this template, the body contains
% some examples of its use.  Look them over.  Then you can
% run LaTeX on this file.  After you have LaTeXed this file then
% you can look over the result either by printing it out with
% dvips or using xdvi.
%
% This template is based on the template for Prof. Sinclair's CS 270.

\documentclass[twoside]{article}
\usepackage{graphicx}
\setlength{\oddsidemargin}{0.25 in}
\setlength{\evensidemargin}{-0.25 in}
\setlength{\topmargin}{-0.6 in}
\setlength{\textwidth}{6.5 in}
\setlength{\textheight}{8.5 in}
\setlength{\headsep}{0.75 in}
\setlength{\parindent}{0 in}
\setlength{\parskip}{0.1 in}

%
% The following commands set up the lecnum (lecture number)
% counter and make various numbering schemes work relative
% to the lecture number.
%
\newcounter{lecnum}
\renewcommand{\thepage}{\thelecnum-\arabic{page}}
\renewcommand{\thesection}{\thelecnum.\arabic{section}}
\renewcommand{\theequation}{\thelecnum.\arabic{equation}}
\renewcommand{\thefigure}{\thelecnum.\arabic{figure}}
\renewcommand{\thetable}{\thelecnum.\arabic{table}}

%
% The following macro is used to generate the header.
%
\newcommand{\lecture}[4]{
   \pagestyle{myheadings}
   \thispagestyle{plain}
   \newpage
   \setcounter{lecnum}{#1}
   \setcounter{page}{1}
   \noindent
   \begin{center}
   \framebox{
      \vbox{\vspace{2mm}
    \hbox to 6.28in { {\bf EE 382C/361C: Multicore Computing
                        \hfill Fall 2016} }
       \vspace{4mm}
       \hbox to 6.28in { {\Large \hfill Lecture #1: #2  \hfill} }
       \vspace{2mm}
       \hbox to 6.28in { {\it Lecturer: #3 \hfill Scribe: #4} }
      \vspace{2mm}}
   }
   \end{center}
   \markboth{Lecture #1: #2}{Lecture #1: #2}
   %{\bf Disclaimer}: {\it These notes have not been subjected to the
   %usual scrutiny reserved for formal publications.  They may be distributed
   %outside this class only with the permission of the Instructor.}
   \vspace*{4mm}
}

%
% Convention for citations is authors' initials followed by the year.
% For example, to cite a paper by Leighton and Maggs you would type
% \cite{LM89}, and to cite a paper by Strassen you would type \cite{S69}.
% (To avoid bibliography problems, for now we redefine the \cite command.)
% Also commands that create a suitable format for the reference list.
\renewcommand{\cite}[1]{[#1]}
\def\beginrefs{\begin{list}%
        {[\arabic{equation}]}{\usecounter{equation}
         \setlength{\leftmargin}{2.0truecm}\setlength{\labelsep}{0.4truecm}%
         \setlength{\labelwidth}{1.6truecm}}}
\def\endrefs{\end{list}}
\def\bibentry#1{\item[\hbox{[#1]}]}

%Use this command for a figure; it puts a figure in wherever you want it.
%usage: \fig{NUMBER}{SPACE-IN-INCHES}{CAPTION}
\newcommand{\fig}[3]{
			\vspace{#2}
			\begin{center}
			Figure \thelecnum.#1:~#3
			\end{center}
	}
% Use these for theorems, lemmas, proofs, etc.
\newtheorem{theorem}{Theorem}[lecnum]
\newtheorem{lemma}[theorem]{Lemma}
\newtheorem{proposition}[theorem]{Proposition}
\newtheorem{claim}[theorem]{Claim}
\newtheorem{corollary}[theorem]{Corollary}
\newtheorem{definition}[theorem]{Definition}
\newenvironment{proof}{{\bf Proof:}}{\hfill\rule{2mm}{2mm}}

% **** IF YOU WANT TO DEFINE ADDITIONAL MACROS FOR YOURSELF, PUT THEM HERE:

\begin{document}
%FILL IN THE RIGHT INFO.
%\lecture{**LECTURE-NUMBER**}{**DATE**}{**LECTURER**}{**SCRIBE**}
\lecture{17}{October 25}{Vijay Garg}{Cassidy Burden}
%\footnotetext{These notes are partially based on those of Nigel Mansell.}

% **** YOUR NOTES GO HERE:

% Some general latex examples and examples making use of the
% macros follow.  
%**** IN GENERAL, BE BRIEF. LONG SCRIBE NOTES, NO MATTER HOW WELL WRITTEN,
%**** ARE NEVER READ BY ANYBODY.

\section{Sorting}

Previously we have talked about three different sorting methods:
\begin{enumerate}
\item Brick Sort
\item Bitonic Sort
\item Merge Sort
\end{enumerate}
We will finish parallel sorting by talking about the parallel merge sort algorithm.

\subsection{Parallel Merge Sort}

Merge sort works by splitting the array into 2 halves, sorting both halves recursively, and then merging the two halves together.

The majority of the work done will be in the merge step. We have previously talked about a work optimal method of merging arrays in parallel by splitting the arrays into chunks and calculating the rank of each element.

We will break the problem into three chunks to allow us to most efficiently distribute our blocks and threads:
\begin{enumerate}
\item Large merges: do parallel scatter merge, assign one block per chunk of scatter
\item Medium sized merges: one merge per block, do parallel scatter merge
\item Small merges: one merge per thread, done sequentially
\end{enumerate}

\centerline{\includegraphics[scale=0.5]{merge_sort.png}}

We eventually end up with a time complexity of $O(\log n)$ and work complexity of ~$O(n \log n)$.

\section{Wait-free Registers}

We now will continue our dicussion on how to build differing levels atomic registers. We will present the following hierarchy in which every register can be built with an unbounded amount of the preceding register.

To recap there are 3 different types of registers that act differently during a concurrent access:
\begin{itemize}
\item Safe - Returns any possible value
\item Regular - Returns value currently being written or previously written value
\item Atomic - Returns values such that history is linearizable
\end{itemize}

\begin{center}
SRSW Safe Bit \\
(store previous value) \\ $\Downarrow$ \\
SRSW Regular Bit \\
(unary construction) \\ $\Downarrow$\\
SRSW Regular Multivalue \\
(timestamp, keep seq number) \\ $\Downarrow$ \\
SRSW Atomic Multivalue \\
(replication, timestamp, communication matrix) \\ $\Downarrow$ \\
MRSW Atomic Multivalue \\
(bakery-style choosing of timestamp) \\ $\Downarrow$ \\
MRMW Atomic Multivalue \\
\end{center}

\subsection{SRSW Regular Bit}

Bits can only have the four histories when looking at the two most recent writes: \\ $(0,0)$ $(0, 1)$ $(1, 0)$ and $(1, 1)$

If we ensure that the only histories are $(0, 1)$ and $(1, 0)$ then our histories will contain all possible values for a bit. Thus when a concurrent access on the safe register occurs, we will either randomly return the value currently being written or the previous value, satisfying the regular register property. To ensure only those two histories occur, we will ignore all duplicate writes to the register.

\subsection{SRSW Regular Multivalue}

To build a regular register than can hold more than just two values, we will have a max size for our register $X$. To represent $X$ values we will need $X$ of our SRSW regular bit registers. We will keep these bits in an array and say the first non-zero bit in the array represents the value of our multivalue register.

\begin{verbatim}
SRSW_Reg_Bit A[X]

read():
  for i = [0 -> X]:
    if A[i] == 1: return i

set(x):
  A[x] = 1
  A[x-1 -> 0] = 0
\end{verbatim}

\subsection{SRSW Atomic Multivalue}

Next we will build an atomic register by attaching a sequence number or timestamp to our value. This is with the assumption that we cannot overflow our sequence number.

\newpage

\subsection{MRSW Atomic Multivalue}

This will be the first register to account for multiple readers. Because we want this register to be atomic we want to make sure that each reader gets the same value once a writer commits a new value.

Because we only have SRSW registers, we will need a register for every reader. The single writer will write to every reader's register, and each reader will read from their own register. However, this is flawed, and can lead to a non-atomic history.

\centerline{\includegraphics[scale=0.5]{MRSW.png}}

The final solution is to have $n^{2}$ SRSW registers, constructing a matrix. We will call this a communication matrix, and it essentially allows each register to share information. When a reader reads from the MRSW register it will want to alert other readers what it just read in order to satisfy atomicity. For example, writing 5 to $comm[i,j]$ is Reader $i$ telling Reader $j$ it just read value 5. Given $n$ readers our MRSW register will work as such:

\begin{verbatim}
SRSW_Atomic_Multi V[n]
SRSW_Atomic_Multi comm[n, n]

set(x):
  V[0 -> n] = x

read(pid):
  val = max_TS(V[pid], comm[0, pid], comm[1, pid], ..., comm[n-1, pid])
  comm[pid, 0 -> pid] = val
  return val
\end{verbatim}

\newpage

\subsection{MRMW Atomic Multivalue}

For $n$ writers we will need $n$ MRSW registers. Because MRSW only allows one writer, each writer will have to write to their own register. Readers will read from whichever register has the highest timestamp.

\centerline{\includegraphics[scale=0.5]{MRMW.png}}

However, because we can, in the worst case, have $n$ writers deciding on a new TS at once, we need an algorithm to atomically decide on a timestamp. Lamport's Bakery has already solved this problem, and we will choose our writer's timestamp in the same way processes choose their tickets in Lamport's Bakery:

\begin{verbatim}
MRSW_Atomic_Multi Regs[n]

set(pid, x):
  ts = max_TS(Regs) + 1
  Regs[pid].set(x, ts, pid) // ts and pid together allow for total ordering

read():
  min_reg = Regs[0]
  for all r in Regs:
    if (min_reg.ts < r.ts) || (min_reg.ts == r.ts && min_reg.pid < r.pid):
      min_reg = r
  return min_reg.read()
\end{verbatim}

\section*{References}
\beginrefs
\bibentry{1}{\sc V.K.~Garg}, Introduction to Multicore Computing
\bibentry{2}{\sc V.K.~Garg}, https://github.com/vijaygarg1/UT-Garg-EE382C-EE361C-Multicore/tree/master/chapter5-wait-free
\endrefs


\end{document}





